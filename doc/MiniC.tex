\documentclass{ctexart}
\usepackage{ctex}
\usepackage[nounderscore]{syntax}
\AtBeginDocument{\catcode`\_=8 }
\ctexset{section/format = {\Large\bfseries}}
\newenvironment{typewriterfont}{\ttfamily}{\par}
\usepackage{minted}
\usepackage{amssymb}
\usepackage{geometry}
\geometry{left=3.0cm, right=3.0cm, top=2.0cm, bottom=2.0cm}

\title{MiniC相关说明}
\date{}
\author{}

\pagestyle{plain}

\begin{document}

\maketitle

\section{概述}
MiniC是C语言的一个子集,对C语言语法进行了大量删减,以产生一种适用于编译实习课程的语言。

MiniC是为了取代原先编译实习课程使用的MiniJava语言而设计的,目的是更好地配合编译原理课程进度,在一定程度上减轻任务量。MiniC基础语法%(我们称之为base语法集)
高度精炼,使同学们无论能力如何,都能完成编译器编写的过程。

同时,对于能力较强的同学,实习课程可以提供一些MiniC语法扩展内容,把从C语言中删除的一些语法加回MiniC,来提升MiniC语言的易用程度和表达能力,对完成扩展内容的同学提供一定程度的加分。

\section{语法描述}
\begin{itemize}
\item 
MiniC取消了C语言中的宏。%即取消了编译预处理过程。
\item 
MiniC中的变量有两种类型,int和一维int数组。%(没有void类型,没有指针)
\item
MiniC中函数返回值只有int,参数可以是int或int数组,程序从main函数开始执行。同时,MiniC不会给函数默认返回值,如果执行完一个函数而没有return,会导致未知行为。

%另外,MiniC中的函数都有return语句,即使返回值无意义,也要以return一个数的方式结束函数。
\item 
单目运算符有\texttt{'!'}和\texttt{'-'}
\item
双目运算符有\texttt{'+','-','*','/','\%','\&\&','||','\textless','\textgreater','==','!='}
\item
合法的表达式参考BNF。
\item 
允许使用函数前置声明(参见样例中\texttt{getint}函数)。
\item 
MiniC程序中允许C风格的单行注释。%和多行注释。
\item 
MiniC只保留\texttt{if-else}条件分支语句和\texttt{while}循环语句。
\item 
为了使MiniC更容易实现,限定\texttt{if,while}后面的括号里和逻辑运算符两边的运算分量只会出现如下两种形式:\\
\indent\ \ \texttt{x>y||(a+b)!=c}这样的逻辑表达式;\\
\indent\ \ \texttt{x}或\texttt{f(x)}或\texttt{a[x]}这样的单个变量或函数。\\
\indent 总之,不会出现类似\texttt{if (a+b)}或\texttt{(b+c)||d}这样的语句。
\end{itemize}
\newpage
\section{BNF}
\setlength{\grammarindent}{8em} % increase separation between LHS/RHS 
\begin{typewriterfont}
\begin{grammar}
<Goal> ::= (VarDefn | FuncDefn | FuncDecl)* MainFunc

<VarDefn> ::= Type Identifier ';'
\alt Type Identifier'['<INTEGER>']' ';'

<VarDecl> ::= Type Identifier
\alt Type Identifier'['<INTEGER>?']'

<FuncDefn>  ::= Type Identifier '(' ( VarDecl ( ',' VarDecl )* )? ')' '\{' (FuncDecl | Statement)* '\}'

<FuncDecl> ::= Type Identifier '(' ( VarDecl (',' VarDecl)*)?')' ';'

<MainFunc> ::= 'int' 'main' '(' ')' '\{' (FuncDecl | Statement)* '\}'

<Type> ::= 'int'

<Statement> ::= '\{' (Statement)* '\}'
\alt 'if' '(' Expression ')' Statement ('else' Statement)?
\alt 'while' '(' Expression ')' Statement
\alt Identifier '=' Expression ';'
\alt Identifier '[' Expression ']' '=' Expression ';'
\alt VarDefn
\alt 'return' Expression ';'

<Expression>	::=	Expression ( '+' | '-' | '*' | '/' | '\%' ) Expression
\alt Expression ( '\&\&' | '||' | '\textless' | '==' | '\textgreater' | '!=' ) Expression
\alt Expression '[' Expression ']'
\alt <INTEGER>
\alt Identifier
\alt ( '!' | '-' ) Expression
\alt Identifier '(' (Identifier (',' Identifier)*)? ')'
\alt '(' Expression ')'

<Identifier>	::=	<IDENTIFIER>

\end{grammar}
\end{typewriterfont}

\newpage
\section{示例}
\small
\begin{minted}[linenos]{c}
int getint(); // 前置函数声明,getint函数是MiniC内置函数,返回一个读入的整数
int putchar(int c); // 内置函数,用于输出字符(参数为ascii码),返回值无意义
                    // 注意!base语法集不包括形如int putchar(int);这种参数名没有具体给出的函数声明
int putint(int i); // 内置函数,用于输出整数,返回值无意义
int getchar(); // 内置函数,返回一个读入的字符的ascii码 (此程序未使用到该函数)
int f(int x) /* 该函数以递归方式计算Fibonacci数 */
{
    if (x < 2) /* if-else语句 */
        return 1;
    else
        return f(x - 1) + f(x - 2); /* 递归函数调用 */
}
int g(int x) /* 该函数以数组和循环语句计算Fibonacci数 */
{
    int a[40]; /* int数组声明
                  注意!base中数组大小必须是常数,不可写成int a[x];或int a[10+30];这样*/
    a[0] = a[1] = 1;
    int i;
    i = 2; /* 注意!base语法集不包括初始化赋值语句  int i = 2; */
    while (i < x + 1) /* while循环是base语法集唯一的循环语句 */
    {
        a[i] = a[i - 1] + a[i - 2];
        ++i;
    }
    return a[x];
}
int n; // 声明了一个全局变量
int main() {
    n = getint();
    if (n < 0 || n > 30) /* 不带else的if语句 */
        return 1;
    putint(f(n));
    putchar(10); // 输出换行符
    putint(g(n));
    putchar(10);
    return 0;
}
\end{minted}
\newpage
\normalsize
\section{MiniC语法扩展}
MiniC设计者们亲身实践了MiniC大部分语法扩展,深切体会到实现一些复杂语法扩展的不宜。

对MiniC语法进行扩展时,应尽量遵循“细致”的原则,避免涉及范围过大的扩展。

比如,整数数据类型扩展:加入8位,16位,32位,64位的有符号和无符号整数。这个扩展涉及的范围就有些大,可以考虑分解成多个扩展:带符号整数扩展、不同长度的整数运算扩展、char字符串扩展(8位有符号整数组成的数组)、字符与字符串表示扩展(加入\texttt{"abc",'\textbackslash n'}这样的表达式)。

具体可以扩展内容,按当然教学通知为准。
\end{document}
