\documentclass{ctexart}
\usepackage{ctex}
\usepackage[nounderscore]{syntax}
\AtBeginDocument{\catcode`\_=8 }
\ctexset{section/format = {\Large\bfseries}}
\newenvironment{typewriterfont}{\ttfamily}{\par}
\usepackage{xcolor}
\usepackage{minted}
\definecolor{bg}{rgb}{0.85,0.85,0.85}
\usepackage{makecell}
\usepackage{listings}
\usepackage{geometry}
\geometry{left=3.0cm, right=3.0cm, top=2.0cm, bottom=2.0cm}

\title{Eeyore相关说明}
\date{}
\author{}

\pagestyle{plain}

\begin{document}
\maketitle

\section{何为Eeyore}
Eeyore是一种三地址码,用作MiniC语法分析后的输出格式。Eeyore的设计同样遵循简洁的原则,使代码易读易调试。

为配合编译原理课程,Eeyore语法实际是龙书中使用的三地址码的扩展版本。

Eeyore在MiniC编译器中的地位,对应着MiniJava编译器中使用到的Piglet和SPiglet两种中间语言。

Eeyore这个名字与Piglet取自同一部动画。

\section{Eeyore语法描述}
Eeyore要求每条语句单独占一行。
\subsection{变量}
Eeyore中的变量有三种:原生变量,临时变量,函数参数。这三种变量分别以\texttt{'T','t','p'}开头,后接一个整数编号,编号从0开始,每个函数单独编号。如\texttt{p0,p1,t0,T0}。

Eeyore的变量声明形如\texttt{var t0}和\texttt{var 8 T0},前者声明了一个int型临时变量,后者声明了有2个元素的原生int数组。

函数参数不需要声明。

除函数参数变量外,其余变量不允许重名。函数内声明的变量作用域为变量声明语句到函数结束语句,函数外声明的变量作用域为变量声明语句到程序最后。

所谓原生变量,是指MiniC中使用的变量转到Eeyore中对应的变量。相应地,临时变量是指MiniC中没有显式对应变量的变量。

用个例子来说明,把左边的MiniC语句翻译到Eeyore:
\begin{table}[H]
    \centering
    \small
    \begin{typewriterfont}
    \begin{tabular}{|c|c|}
        \hline
        MiniC & Eeyore \\
        \hline
        \makecell[l]{int a;\\ int b;\\ int c;\\a = b + 2 * c;} & \makecell[l]{var T0\\var T1\\var T2\\var t0\\t0 = 2 * T2\\var t1\\t1 = T1 + t0\\T0 = t1} \\
        \hline
    \end{tabular}
    \end{typewriterfont}
\end{table}

上面\texttt{T0,T1,T2}是原生变量,分别对应MiniC中的\texttt{a,b,c}。\texttt{t0,t1}是临时变量,分别对应中间运算结果\texttt{2*c, b+2*c}。

其实这两种变量在语义上没必要做如此区分,Eeyore区分二者是为了方便用户调试。

\subsection{表达式}
由上面的例子可以发现,Eeyore的表达式很易读。

因为是三地址码,每条表达式语句最多涉及三个变量。

允许直接把整数用作运算分量。

Eeyore表达式支持的单目运算符有\texttt{'!','-'}

支持的双目运算符有\texttt{'!=','==','\textgreater'.'\textless','\&\&','||','+','-','*','/','\%'},前6个是逻辑运算符。

数组操作语句形如\texttt{T0 [t0] = t1}和\texttt{t0 = T0 [t1]}。

注意!因为MiniC的base语法集只有int和int数组类型,数组操作语句中括号内的数应当是4的倍数。

注意!如果\texttt{T0}是数组,那\texttt{t0 = T0 + 4}这样的语句实际上是指针运算,\texttt{t0}的类型是int不是数组。

\subsection{函数}
Eeyore中的函数以\texttt{'f\_'}开头,后接函数名,如\texttt{f\_main,f\_getint}。

函数定义语句形如\texttt{f\_putint [1]},中括号内的整数表示该函数的参数个数,函数结束处应有函数结束语句,形如\texttt{end f\_xxx}。

函数外的变量声明语句被视为全局变量声明,函数内的视为局部变量声明。

函数调用语句形如\texttt{t0 = call f\_xxx}。如果该函数有参数,还需要在调用语句前插入传参语句,形如\texttt{param t1},每个传参语句只传一个参数,如有多个参数,应按照参数顺序一个一个传。

函数返回语句形如\texttt{return t0}。

\subsection{标号与跳转}
Eeyore中的标号以小写字母\texttt{'l'}开头,后接整数编号,编号从0开始,如\texttt{l0,l1}。标号用来指明跳转语句的跳转地点,标号声明语句形如\texttt{l0:}。

跳转语句分两种:无条件跳转、条件跳转。如\texttt{goto l1}和\texttt{if t0 < 1 goto l0}。

不允许跨函数跳转。

\newpage
\section{BNF}
\begin{typewriterfont}
\setlength{\grammarindent}{8em} % increase separation between LHS/RHS 
\begin{grammar}
\small
<Declaration> ::= 'var' <INTEGER>? Variable

<FunctionDecl> ::= Function '[' <INTEGER> ']' '\textbackslash n' ((Expression | Declaration)'\textbackslash n')* 'end' Function
%t1 = t1 <op>/<relop> t2
%t1 = <op> t2
%t1 = <type> t2 (类型转换)
%t1 = t2
%t1 [ t2 ] = t3
%* t1 = t2
%t1 = t2 [ t3 ]

<RightValue> ::= Varibale | <INTEGER>

<Expression>	::=	Variable '=' RightValue OP2 RightValue
\alt Variable '=' OP1 RightValue
\alt Variable '=' RightValue
\alt Variable '[' RightValue ']' = RightValue
\alt Variable = Variable '[' RightValue ']'
\alt 'if' RightValue LogicalOP RightValue 'goto' Label
\alt 'goto' Label
\alt Label ':'
\alt 'param' RightValue
\alt Variable '=' 'call' Function
\alt 'return' RightValue


<Identifier>	::=	<IDENTIFIER>

<Variable> ::= <VARIABLE>

<Label> ::= <LABEL>

<Function> ::= <FUNCTION>

\end{grammar}
\end{typewriterfont}
\newpage
\section{示例}
\begin{table}[H]
    \centering
    \small
    \begin{typewriterfont}
    \begin{tabular}{|c|c|}
        \hline
        MiniC & Eeyore \\
        \hline
        \makecell[l]{
int getint();\\
int putint(int x);\\
int n;\\
int a[10];\\
int main()\\
\{\\
\ \ n = getint();\\
\ \ if (n > 10)\\
\ \ \ \ return 1;\\
\ \ int s;\\
\ \ int i;\\
\ \ i = 0;\\
\ \ s = i;\\
\ \ while (i < n) \{\\
\ \ \ \ a[i] = getint();\\
\ \ \ \ s = s + a[i];\\
\ \ \ \ ++i;\\
\ \ \}\\
\ \ putint(s);\\
\ \ return 0;\\
\}
        } & \makecell[l]{
var T0\\
var 40 T1\\
f\_main [0]\\
T0 = call f\_getint\\
var t0\\
t0 = T0 - 10\\
var t1\\
t1 = t0 > 0\\
if t1 == 0 goto l0\\
return 1\\
l0:\\
var T2\\
var T3\\
T3 = 0\\
T2 = T3\\
l1:\\
var t2\\
t2 = T3 < T0\\
if t2 == 0 goto l2\\
var t3\\
t3 = 4 * T3\\
var t4\\
t4 = call f\_getint\\
T1 [t3] = t4\\
var t5\\
t5 = T1 [t3]\\
T2 = T2 + t5\\
T3 = T3 + 1\\
l2:\\
param T2\\
call f\_putint\\
return 0\\
end f\_main
} \\
        \hline
    \end{tabular}
    \end{typewriterfont}
\end{table}

\newpage
\section{Eeyore模拟器使用方式}
\begin{typewriterfont}
\begin{lstlisting}
Usage ./Eeyore [-d] <filename>

-d : enable debug mode

- e.g. ./Eeyore -d test.in

出现"> "提示符表示进入debug模式,支持如下指令: 

+ p <pc/symbol/label/funciton name>     
    - e.g. p pc, p c1, p t1
    - Print the value of the symbol
+ s <number>
    - e.g. s 10
    - Run n Step
+ n
    - e.g. n
    - Run 1 Step
+ u <number/function/label>
    - e.g. u 10, u f_g, u l1
    - Run until pc equal to number or until the function or label.
+ r
    - e.g. r
    - Disable the debug mode and run until the program exit.
\end{lstlisting}
\end{typewriterfont}
\end{document}
